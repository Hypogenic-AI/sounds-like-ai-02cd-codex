\section{Results}
\label{sec:results}
\para{Probe accuracy.} Linear probes separate AI vs human answers with high accuracy at all tested layers (\Tabref{tab:probe_results}). The best layer is 12 with 97.8\% accuracy and perfect precision on the test split, indicating strong linear separability of AI-sounding style in mid-layer residual activations.

\begin{table}[t]
    \centering
    \resizebox{0.85\textwidth}{!}{%
    \begin{tabular}{lcccc}
        \toprule
        Layer & Accuracy & Precision & Recall & F1 \\
        \midrule
        0  & 0.944 & 0.935 & {\bf 0.956} & 0.945 \\
        12 & {\bf 0.978} & {\bf 1.000} & {\bf 0.956} & {\bf 0.977} \\
        23 & 0.967 & 0.977 & {\bf 0.956} & 0.966 \\
        \bottomrule
    \end{tabular}%
    }
    \caption{Probe performance on the \hcThree test split (90 examples). Best values are in bold; recall is tied across layers.}
    \label{tab:probe_results}
\end{table}


\para{Steering effects.} Adding the mean-difference direction increases probe-based AI-likeness, while removing it decreases AI-likeness (\Tabref{tab:steering_results}). The mean changes relative to \baseline are +0.048 (AI-plus) and -0.156 (AI-minus). A two-sided paired t-test yields $p=0.463$ for AI-plus vs base and $p=0.078$ for AI-minus vs base, which is not significant at conventional thresholds in this small sample (10 prompts).

\begin{table}[t]
    \centering
    \begin{tabular}{lcc}
        \toprule
        Condition & Mean AI-likeness & Std \\
        \midrule
        Base    & 0.655 & 0.224 \\
        AI-plus & {\bf 0.704} & {\bf 0.108} \\
        AI-minus & 0.499 & 0.255 \\
        \bottomrule
    \end{tabular}
    \caption{Probe-based AI-likeness scores for 10 prompts under steering at layer 12. Higher scores indicate more AI-like style.}
    \label{tab:steering_results}
\end{table}


\para{Score distributions.} \Figref{fig:score_distributions} shows the distribution of probe scores across conditions. AI-minus exhibits a lower mean and higher variance, which aligns with observed reductions in AI-like style but also hints at increased instability.

\begin{figure}[t]
    \centering
    \includegraphics[width=0.95\linewidth]{figures/score_distributions.png}
    \caption{Distribution of probe-based AI-likeness scores across base, AI-plus, and AI-minus generations (10 prompts).}
    \label{fig:score_distributions}
\end{figure}

\para{Judge ratings.} On the same 10 prompts, \gptfourone ratings show small shifts in the expected direction (mean $\Delta=+0.10$ for AI-plus, $\Delta=-0.40$ for AI-minus), with $p=0.343$ and $p=0.223$ respectively, again underpowered in this pilot.
